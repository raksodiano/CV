\documentclass[11pt,a4paper]{moderncv}

% moderncv themes
% colors
% black, blue, burgundy, green, grey, orange, purple, red
% Themes
% banking, classic, casual, oldstyle, fancy
\moderncvtheme[blue]{banking}

% character encoding
\usepackage[utf8]{inputenc}
\renewcommand{\familydefault}{\sfdefault}

% adjust page margins
\usepackage[scale=0.9]{geometry}
\recomputelengths

% for placing content in footer/header
\usepackage{fancyhdr}
\pagestyle{fancy}
\pagenumbering{gobble}

% personal data
\firstname{Oskar}
\familyname{Torres}
\title{Ingeniero en Informática}
\address{Medellín}{Colombia}
\phone[mobile]{(+57) 311 332 81 63}
\email{oskar.torres.1234@gmail.com}
\homepage{blog-un-dev-mas-7f0869.gitlab.io}
\social[linkedin]{oskartorresojeda}
\social[github]{raksodiano}

\begin{document}
\maketitle

\section{\textbf{Perfil Laboral}}
Ingeniero en informática, con más 10 años de experiencia en cargos como Desarrollador Backend y Desarrollador Fullstack. Desempeñando funciones como planificación, desarrollo, optimización y despliegues automatizados. Con conocimientos en gestión y planificación proyectos y habilidades en JavaScript, Python, Node.js, Vue.js e Inteligencia Artificial. Con competencias laborales en adaptabilidad al cambio, orientación al logro, pensamiento analítico, trabajo colaborativo y apertura al aprendizaje.

\section{\textbf{Habilidades técnicas}}

\cvlistitem{\textbf{Frameworks}: NestJS, Laravel, Symfony, Vue.js.}
\cvlistitem{\textbf{Lenguajes}: Javascript, Typescript, PHP, Python.}
\cvlistitem{\textbf{Base de datos}: PostgreSQL, MariaDB, MongoDB, Firebase.}
\cvlistitem{\textbf{Nube}: Google Cloud Platform.}
\cvlistitem{\textbf{Contenedores}: Docker.}
\cvlistitem{\textbf{Web Servers}: Apache, NGINX.}
\cvlistitem{\textbf{Sistemas Operativos}: Debian, Ubuntu, AlmaLinux, RockyLinux.}
\cvlistitem{\textbf{Control de versiones}: Git}

\section{\textbf{Experiencia}}
\cventry{2023 -- Actualidad}{Desarrollador Fullstack}{Instituto Pacífico}{Lima, Perú}{Remoto}{Responsable de la modernización y optimización, con enfoque en escalabilidad, rendimiento y buenas prácticas de desarrollo.
\begin{itemize}%
\item Lideré la migración de una arquitectura monolítica a microservicios, mejorando la mantenibilidad y escalabilidad del sistema.
\item Implementé un sistema de despliegues automáticos, reduciendo errores manuales y acelerando el ciclo de desarrollo.
\item Reestructuré la arquitectura de más de 30 servicios, lo que resultó en una plataforma más robusta y eficiente.
\item Optimizé el consumo del servidor, logrando una reducción significativa en los recursos utilizados por los servicios activos.
\end{itemize}}

\cventry{2022 -- 2023}{Desarrollador Backend}{Seeed}{USA, Miami}{Remoto}{Contribuí al diseño y desarrollo de sistemas y módulos multifuncionales orientados a escalar operaciones internas y mejorar la eficiencia de procesos clave.
\begin{itemize}
\item Lideré la planificación y desarrollo de módulos backend en NestJS, reduciendo los tiempos de desarrollo en un 30\%.
\item Implementé arquitectura basada en microservicios, mejorando la escalabilidad y reduciendo errores de integración.
\item Desarrollé e integré funciones en Google Cloud, optimizando el rendimiento de procesos críticos.
\item Diseñé soluciones con Auth0 para la gestión de autenticación segura y con Square para procesamiento de pagos, logrando una integración confiable y fluida.
\end{itemize}}

\cventry{2020}{Desarrollador Full Stack}{ePayco}{Medellín, Colombia}{Remoto}{Contribuí al desarrollo de módulos clave para la pasarela de pagos, enfocado en mejorar la eficiencia y la experiencia del usuario.
\begin{itemize}%
\item Desarrollé y optimicé módulos multifuncionales, mejorando el rendimiento y la estabilidad del sistema de pagos.
\item Colaboré en la refactorización del frontend y backend de componentes críticos, reduciendo la deuda técnica y mejorando la mantenibilidad.
\end{itemize}}

\section{\textbf{Formaci\'on acad\'emica}}
\cventry{2012 -- 2019}{Ingeniero en Informática}{U.N.E.R.G}{Venezuela}{}{}

% \section{Idiomas}
% \cvitem{Espa\~nol}{Nativo}
% \cvitem{Ingles}{Principiante(A2)}

\end{document}
